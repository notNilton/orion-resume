%-------------------------
% Currículo em Latex
%-------------------------
%-------------------------
% Configuração do Currículo em Latex
%-------------------------

%---- Pacotes e Funções Necessárias ----

\documentclass[a4paper,11pt]{article}
\usepackage{latexsym}
\usepackage{xcolor}
\usepackage{float}
\usepackage{ragged2e}
\usepackage[empty]{fullpage}
\usepackage{wrapfig}
\usepackage{lipsum}
\usepackage{tabularx}
\usepackage{titlesec}
\usepackage{geometry}
\usepackage{marvosym}
\usepackage{verbatim}
\usepackage{enumitem}
\usepackage[hidelinks]{hyperref}
\usepackage{fancyhdr}
\usepackage{fontawesome5}
\usepackage{multicol}
\usepackage{graphicx}
\usepackage{cfr-lm}
\usepackage[T1]{fontenc}

\setlength{\multicolsep}{0pt}
\pagestyle{fancy}
\fancyhf{} % limpar todos os campos de cabeçalho e rodapé
\fancyfoot{}
\renewcommand{\headrulewidth}{0pt}
\renewcommand{\footrulewidth}{0pt}
\geometry{left=1.4cm, top=0.8cm, right=1.2cm, bottom=1cm}

\usepackage[most]{tcolorbox}
\tcbset{
    frame code={}
    center title,
    left=0pt,
    right=0pt,
    top=0pt,
    bottom=0pt,
    colback=gray!20,
    colframe=white,
    width=\dimexpr\textwidth\relax,
    enlarge left by=-2mm,
    boxsep=4pt,
    arc=0pt, outer arc=0pt,
}

\urlstyle{same}
\raggedright
\setlength{\tabcolsep}{0in}

% Formatação de Seções
\titleformat{\section}{
  \vspace{-4pt}\scshape\raggedright\large
}{}{0em}{}[\color{black}\titlerule \vspace{-7pt}]

%-------------------------
% Comandos personalizados
\newcommand{\resumeItem}[2]{
  \item{
    \textbf{#1}{\hspace{0.5mm}#2 \vspace{-0.5mm}}
  }
}

\newcommand{\resumePOR}[3]{
\vspace{0.5mm}\item
    \begin{tabular*}{0.97\textwidth}[t]{l@{\extracolsep{\fill}}r}
        \textbf{#1}\hspace{0.3mm}#2 & \textit{\small{#3}} 
    \end{tabular*}
    \vspace{-2mm}
}

\newcommand{\resumeSubheading}[4]{
\vspace{0.5mm}\item
    \begin{tabular*}{0.98\textwidth}[t]{l@{\extracolsep{\fill}}r}
        \textbf{#1} & \textit{\footnotesize{#4}} \\
        \textit{\footnotesize{#3}} &  \footnotesize{#2}\\
    \end{tabular*}
    \vspace{-2.4mm}
}

\newcommand{\resumeProject}[4]{
\vspace{0.5mm}\item
    \begin{tabular*}{0.98\textwidth}[t]{l@{\extracolsep{\fill}}r}
        \textbf{#1} & \textit{\footnotesize{#3}} \\
        \footnotesize{\textit{#2}} & \footnotesize{#4}
    \end{tabular*}
    \vspace{-2.4mm}
}

\newcommand{\resumeSubItem}[2]{\resumeItem{#1}{#2}\vspace{-4pt}}
% \renewcommand{\labelitemii}{$\circ$}
\renewcommand{\labelitemi}{$\vcenter{\hbox{\tiny$\bullet$}}$}
\newcommand{\resumeSubHeadingListStart}{\begin{itemize}[leftmargin=*,labelsep=0mm]}
\newcommand{\resumeHeadingSkillStart}{\begin{itemize}[leftmargin=*,itemsep=1.7mm, rightmargin=2ex]}
\newcommand{\resumeItemListStart}{\begin{justify}\begin{itemize}[leftmargin=3ex, rightmargin=2ex, noitemsep,labelsep=1.2mm,itemsep=0mm]\small}
\newcommand{\resumeSubHeadingListEnd}{\end{itemize}\vspace{2mm}}
\newcommand{\resumeHeadingSkillEnd}{\end{itemize}\vspace{-2mm}}
\newcommand{\resumeItemListEnd}{\end{itemize}\end{justify}\vspace{-2mm}}
\newcommand{\cvsection}[1]{%
\vspace{2mm}
\begin{tcolorbox}
    \textbf{\large #1}
\end{tcolorbox}
    \vspace{-4mm}
}
\newcolumntype{L}{>{\raggedright\arraybackslash}X}%
\newcolumntype{R}{>{\raggedleft\arraybackslash}X}%
\newcolumntype{C}{>{\centering\arraybackslash}X}%
%---- Fim dos Pacotes e Funções ------
 % Inclui o arquivo de configuração

%---- Definir Informações do Currículo ----

\newcommand{\name}{Nilton Aguiar dos Santos} % Meu Nome
\newcommand{\course}{Engenharia de Computação} % Meu Curso
\newcommand{\phone}{+55 (65) 99278-5635} % Meu Número de Telefone
\newcommand{\emaila}{nilton.naab@gmail.com} % Meu Email
\newcommand{\github}{https://github.com/notNilton} % Meu Github
\newcommand{\linkedin}{https://www.linkedin.com/in/notnilton/} % Meu Linkedin
\newcommand{\portifolio}{https://github.com/notNilton} % Meu Portifolio


\begin{document}
\fontfamily{cmr}\selectfont

%----------CABEÇALHO-----------------
{
\begin{tabularx}{\linewidth}{L r} \\
    \textbf{\Large \name} & {\raisebox{0.0\height}{\footnotesize \faPhone}\ \phone}
    \\
    {Engenheiro de Software}
    &
    \href{\portifolio}{\raisebox{0.0\height}{\footnotesize \faGlobe}\ {Portfólio}}
    \\
    \href{mailto:\emaila} {\raisebox{0.0\height}{\footnotesize \faEnvelope}\ {\emaila}}
    \href{\linkedin}{\raisebox{0.0\height}{\footnotesize \faLinkedin}\ {LinkedIn} }
    \href{\github}{\raisebox{0.0\height}{\footnotesize \faGithub}\ {GitHub} }
\end{tabularx}
} 
\vspace{1mm}

Graduando em Engenharia de Computação, com vivência nas áreas de Front-end, Back-end, Banco de Dados e gerenciamento de projetos, focado em alcançar resultados eficientes e de alta qualidade.

\section{\textbf{Educação}}
    \resumeSubHeadingListStart
    \resumeSubheading
    {Bacharelado em Engenharia de Computação}{}
    {Universidade Federal de Mato Grosso, Brasil}{2020-25}
    \resumeSubHeadingListEnd
\vspace{-5.5mm}

%-----------PROJETOS PESSOAIS-----------------
\section{\textbf{Projetos Pessoais}}
\resumeSubHeadingListStart
    \resumeProject
      { RADARE - \textit{Reconciliation and Data Analysis in a Responsive Environment}}
      {Aplicativo web baseado em React para reconciliação de dados.}
      {}

      \resumeItemListStart
        \item Reconciliação de dados industriais utilizando minimização de funções multivariáveis pelo método dos multiplicadores de Lagrange.
        \item {Capaz de reconciliar dados em grande escala (milhões de registros), tanto em tempo real quanto em \textit{batches}, identificando e corrigindo erros nos dados por meio da reconciliação e ajustando as variáveis para seus valores corretos.}
        \item {Client construído em \textbf{Typescript} e \textbf{React} para melhor responsividade, com a biblioteca \textbf{ReactFlow} para interação com os nódulos.}
        \item {Servidor desenvolvido com \textit{endpoints} de API, facilitando a integração com outras ferramentas e sistemas.}
        \item {Banco de dados altamente escalável e robusto, implementado com \textbf{PostgreSQL} e \textbf{TimescaleDB} para armazenamento eficiente de séries temporais.}
        \item {Tecnologias Utilizadas: Python (Flask), TypeScript (React), PostgreSQL (TimescaleDB).}
      \resumeItemListEnd

       \resumeProject
      { AAICAP - \textit{Artificial Intelligence Correction and Augmentation Pipeline}}
      {Projeto baseado em Python para correção de imagens pixel-art geradas por Inteligência Artificial.}
      {}

      \resumeItemListStart
        \item {Utiliza técnicas de tratamento de imagens para corrigir falhas geradas em imagens pixel-art.}
        \item {Capaz de otimizar a qualidade das imagens enquanto mantém características relevantes, ajustando dinamicamente os parâmetros de processamento.}
        \item {Pipeline de correção e aprimoramento desenhado para processamento eficiente de grandes volumes de imagens, com suporte para automação e API.}
        
        \item {Tecnologias Utilizadas: Python, Flask, OpenCV, PIL.}
      \resumeItemListEnd
      
  \resumeSubHeadingListEnd
\vspace{-8.5mm}

\section{\textbf{Experiência}}
  \resumeSubHeadingListStart
    \resumeSubheading
      { Estágio Desenvolvimento Back-end}{Presencial}
      {SEMA-MT}{Maio 2021 - Maio 2022}
      \vspace{-2.0mm}
      \resumeItemListStart
        \item {Contribuí no desenvolvimento de uma ferramenta de campo utilizando \textbf{C\#} e \textbf{.NET}, projetada para coletar e gerenciar dados em ambientes remotos.}
        \item {Implementei a funcionalidade de sincronização com banco de dados, garantindo a consistência dos dados tanto em modos online quanto offline.}
        \item {Desenvolvi e otimizei a capacidade de funcionalidade offline, permitindo que os usuários continuassem a utilizar a ferramenta sem conectividade, com a sincronização automática ao restabelecer a conexão.}
        \item {Colaborei com a equipe na integração da ferramenta com sistemas internos e externos por meio de APIs robustas.}
        \item {\textbf{Tecnologias Utilizadas:} C\#, .NET, SQL Server, APIs RESTful.}
      \resumeItemListEnd


    \vspace{-3.0mm}
    
        \resumeSubheading
      { Desenvolvedor Junior Full Stack}{Presencial}
      {PGE-MT}{Maio 2024 - Atual}
      \vspace{-2.0mm}
      \resumeItemListStart
        \item {Implementei automações com \textbf{Google Apps Script} e \textbf{AppSheet}, otimizando fluxos internos.}
        \item {Desenvolvi sistemas de gerenciamento de processos jurídicos e controle de escalas para procuradores, usando \textbf{TypeScript} e \textbf{React} no front-end, \textbf{PostgreSQL} como banco de dados e deploy na \textbf{Firebase} com autenticação via \textbf{Google Authentication}.}
        \item {Integrei o \textbf{ChatGPT} para resumos automáticos de processos e pré-avaliação de litispendência, utilizando \textbf{Node.js} e a \textbf{API} do ChatGPT.}
        \item {\textbf{Tecnologias Utilizadas:} TypeScript, React, PostgreSQL, Firebase, Node.js, Google Apps Script, AppSheet, ChatGPT API.}
      \resumeItemListEnd

    \vspace{-3.0mm}
  \resumeSubHeadingListEnd
  

\section{\textbf{Habilidades Técnicas e Interesses}}
\begin{itemize}[leftmargin=0.05in, label={}]
    \small{\item{
     \textbf{Idiomas:}{ Português (nativo), Inglês (fluente), Espanhol (mediano), Francês (iniciante) e Alemão (iniciante).} \\
     \textbf{Linguagens:}{ C/C++, Rust, Python e TypeScript. } \\
     \textbf{Bibliotecas:}{ NumPy, Pandas, ReactJS e Redux. }\\ 
     \textbf{Ferramentas:}{ Node.js, Git, GitHub,  Gitlab, AppScript e AppSheet. } \\ 
     \textbf{Frameworks:}{ ReactJS, Flask, Express.js e FastAPI. } \\
     \textbf{Nuvem/Bancos de Dados:}{ Firebase, PostgreSQL e TimescaleDB. } \\  
    }}
\end{itemize}

\end{document}
